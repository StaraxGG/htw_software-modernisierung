\section{On-premise in-hose vs Cloud Computing bei Providern}
\label{VgmOnPremise}
%=============================


Um auch Cloud Computing, von einer im Unternehmen entwickelten und betriebenen Software zu unterscheiden (On-premise in-house), werden im Folgenden die Punkte Deployment, Kontrolle, (Daten-)Sicherheit und (IT-)Compliance Anforderungen voneinander abgegrenzt. Es soll deutlich gemacht werden, welche Vor- und Nachteile die Unternehmen im Zuge der Überlegungen für die Software-Modernisierung betrachten sollen und welche Restriktionen  beeinflussen. On-Premise Software ist auf den Unternehmenservern installiert. Der Zugriff ist, neben der Firewall, von verschiedene in-house Faktoren abhängig. Diese Applikationen gelten als zuverlässig und sicher und ermöglichen den Unternehmen, ein hohes Maß an Kontrolle. \cite{fisher2018cloud}
\\\\
Folgend werden die beiden Lösungsvarianten unter verschiedenen Gesichtspunkten voneinander abgegrenzt.
\begin{description}   
 \item [Deplyoment] \hfill \\
In einer on-premise Enviroment findet das Deployment innerhalb der Unternehmen-IT-Infrastruktur statt. Entsprechend sind die Verantwortlich für die Wartung der Lösung und alle mit ihr verbundenen Prozesse. \\Die virtuelle Cloud-Infrastruktur hingegen bietet hohe Flexibilität bei der Implementation auf einer  breiten Infrastruktur. Zusätzlich ermöglicht es eine schnellere Installation und Support-Services, was auch die Systemadministrator effizient unterstützen kann.\cite{bibi2012business}
 \item [Kontrolle] \hfill \\
Die Unternehmen mit einer on-premise Lösungen erheben, verfügen, analysieren und evaluieren die Unternehmen alle Daten. Was besonders für die Verwendung einer on-premise Lösung für Unternehmen spricht, die in einer stark-regulierten Branche arbeiten.\\
In der Cloud-Computing-Umgebung stellen sich viele Unternehmen und Provider die Frage nach des Eigentums der Daten. Diese befinden sich physisch bei den Drittanbietern. Falls es zu Problemen und Ausfällen kommt, können Unternehmen möglicherweise nicht auf diese Daten zugreifen.\cite{bibi2012business}
 \item [(Daten-)Sicherheit] \hfill \\
Unternehmen mit besonders sensiblen Informationen, wie Behörden und Banken, müssen über ein gewisses Maß an Sicherheit und Datenschutz verfügen, das eine on-premise in-house Lösung bietet.
Sicherheitsbedenken sind im Allgemeinen auch bei durch Drittanbieter angebotenen Cloud Computing-Lösungen ein Problem. Auch, wenn die Provider im Laufe der Zeit die Robustheit der gesamten Infrastruktur unter Beweis gestellt haben. \cite{klotz2008compliance} \cite{falk2012compliance}
 \item [Compliance] \hfill \\
Viele Unternehmen arbeiten, unabhängig von der Branche, unter behördlichen Kontrollen. Für Unternehmen, die solchen Vorschriften unterliegen, ist es zwingend erforderlich, dass sie ihre (IT-)Compliance permanent im Auge haben. Hierfür eignet sich die on-premise Variante eher. \\
Auch dann sind  Unternehmen für die Einhaltung von (IT-)Compliance Richtlinien verantwortlich, wenn sie eine Cloud-Lösung bei einem Drittanbieter haben. Sensible Daten müssen geschützt werden um Kunden/Kundinnen, Partner/Partnerinnen und Mitarbeiter/Mitarbeiterinnen  ihre Privatsphäre gewährleisten zu können.\cite{klotz2008compliance}
\end{description}
