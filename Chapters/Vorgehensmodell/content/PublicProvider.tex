\section{Bekannte Cloud-Computing Anbieter}
\label{VgmPublicProvider}
%=============================
Die Entscheidung bei welchem Anbieter Unternehmen ihre Cloud-Lösungen beziehen wollen, spielt  eine wichtige Rolle im Modernisierungsprozess. Technik-Giganten wie Amazon mit \textit{Amazon Web Services (AWS)}, Microsoft mit \textit{Azure} und Google mit \textit{Google Cloud }bieten mittlerweile schon länger Cloud-Lösungen für Unternehmen, als auch Privatpersonen an. Sie bieten alle die in Kapitel \ref{VgmAsAService} \nameref{VgmAsAService} vorgestellten Arten \acs{IaaS}, \acs{PaaS}, \acs{SaaS}. 
Interessante Gegenüberstellungen wie Bedienbarkeit, Geschwindigkeit und Zuverlässigkeit lassen sich in der Literatur, als auch von den Anbietern keine finden. Die für Kunden merkbaren Unterschiede stellen daher nur die Anbieterspezifischen Zusatzvorteilen, wie zum Beispiel Tools, Funktionen und Werkzeuge dar. 




