\section{Vorab}
\label{VgmVorab}
%=============================

Auf den in den vorigen Teilen beschriebenen Grundlagen, Techniken und Vorteilen von Software-Modernisierung aufbauend wird folgend mit Cloud Computing ein Praxisbeispiel behandelt.
Abhängig der jeweiligen Branche greifen heutzutage immer mehr Unternehmen auf Cloud Lösungen von Drittanbietern(Provider) zurück. Diese bieten den Unternehemen, neben der Angebotsvielfalt, einige qualitative und quantitative Benefits. Daher erfreut sich deren Verwendung heutzutage immer größer werdender popularität und Beliebtheit \cite{skyhigh}. Die Entscheidung der Betrachtung für Cloud Computing in dieser Ausarbeitung erfolgte nach keinem  Ranking, das Cloud Computing im Vergleich von anderen Technologien hervorhebt. Der Einsatz dieser Lösung in der Praxis richtet sich nach den Bedürfnissen und Zielen der Unternehmen. Für die weiteren Betrachtungen wird von Unternehmen in der freien Marktwirtschaft ausgegangen. 
\\\\
Als exemplarische, moderne, zukunftsorientierte Technologie liefert Cloud Computing Grund genug eine facettenreiche  Betrachtung  einer Software-Modernisierungsmaßnahme durchzuführen. Die entstehenden zahlreichen Fragestellung, Analysen, Auswertungen und Evaluationen beeinflussen maßgeblich die zukünftige IT-Landschaft eines Unternehmens. 
\\\\
Eine Modernisierungsmaßnahme, bei der ein Legacy-System zu einer künftigen Cloud-Lösung entwickelt werden soll, erfordert gewisse Grundlagen. Neben der Definition des \textit{National Institut of Standards and Technologies} zu Cloud Computing, wird  notwendige Wissen in Abschnitt \ref{VgmCC} \nameref{VgmCC}  vermittelt.  Anschließend wird in Kapitel  \ref{VgmOnPremise} \nameref{VgmOnPremise} die Lösungsmöglichkeit von einer „on-premise in house-Lösung“ zu Cloud Computing abgegrenzt. Die konkrete Durchführung einer Software-Modernisierungsmaßnahme stellt für die  Unternehmen ein kompliziertes und komplexes Projekt dar. Zur Weiteren Vermittlung wird in Abschnitt \ref{VgmVorgehensmodell} \nameref{VgmVorgehensmodell} eine vereinfachte Roadmap, also ein Vorgehensmodell zur Umsetzung einer Modernisierungsmaßnahme einer Cloud Computing Lösung vorgestellt. Die Möglichkeiten wie Services, beziehungsweise Dienste, in der späteren Cloud angeboten werden können, wird in \ref{VgmAsAService} \nameref{VgmAsAService} voneinander abgrenzt. In \ref{VgmPublicProvider} \nameref{VgmPublicProvider}  schließt sich der Kreis, mit der abschließenden Vorstellungen von Amazons „Amazon web services (aws)“ und Microsofts „Azure“.
\newpage