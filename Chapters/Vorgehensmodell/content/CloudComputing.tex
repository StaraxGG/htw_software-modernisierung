\section{Cloud Computing im Überblick}
\label{VgmCC}
%=============================
Für einen gemeinsamen Kontext zu dem Begriff Cloud Computing wird die Definition des National Institut of Standards und Technology im Rahmen dieser Ausarbeitung verwendet. Dieses führt fünf grundlegende und einheitliche Charakterstika für Cloud Computing Systeme ein. Die zusammenfassenden Oberbegriffe dieser sind der On-demand self service, Broad network access, Resource pooling, die Rapid elasticity und dem Measured service.
\\\\
Im Wesentlichen fassen die genannten Punkte die Möglichkeiten zusammen für einen entfernten, einseitigen Zugriff aus dem Netz mit heterogenen Endgeräten auf Rechen- und Serverressourcen. Diese Ressourcen werden, in Abhängigkeit der Bedürfnisse und Notwendigkeit, skalierbar und bedarfsgerecht, nach dem Mehrmandantenprinzip bereitgestellt. Die Benutzer und Benutzerinnen sehen dabei nicht, mit welchen (physischen) Kapazitäten die Leistung erbracht wird. Der measured Service beschreibt die Anforderung für die Bereitstellung erhobener Daten. Diese schließen unter anderem Kennzahlen wie die Zugriffszeiten, Anzahl aktiver Nutzer und Nutzerinnen, Server- und Speicherauslastung erhöhen die Transparenz mit ein. Dies ist nicht nur für die Transparenz seitens der Unternehmen wichtig, die für diese Dienste bezahlen, sondern auch für die Drittanbieter dieser Lösungen, die somit ihre Ressourcen überwachen und gegebenenfalls mit physischer Hardware der erwartbaren Nachfrage aufrüsten müssen. \cite{nist}
\\\\

Cloud Computing basiert auf der \ac{SOA} \cite{CcIBM}, also der Service-, beziehungsweise Dienst-Orientierten Architektur. Ein Dienst/Service ist eine in sich geschlossene Softwareeinheit, die mit Anwendungen und/oder anderen Diensten über einen lose gekoppelten, oft asynchronen Kommunikationskanal kommuniziert. \acs{SOA} bezeichnet dabei die gesamte Architektur, bestehend aus den miteinander kommunizierenden Diensten. Grundlegende Eigenschaften der \acs{SOA} ist ein einheitlicher Kommunikationskanal und die lose gekoppelten Dienste. Bei deren Entwicklung liegt der Fokus auf einem qualitativ hochwertigen, bereitgestellten Dienst. \cite{lewis2005service} \\\\

Dies ist nicht ohnehin von besonderer Bedeutung, da man Fehler aus der Vergangenheit (Beispielweise. schlechte Code-Entwicklung, schlechtere Services, etc.) nicht im neuen System wiederholen möchte.  \acs{SOA} bietet somit in seinen Grundzügen einen soliden Kern, auf dem Cloud Computing aufbaut. 
\\\\
Neben der Bestimmung der quantitativen und qualitativen Vor- und Nachteile, spielen oft weitere Faktoren eine wichtige Rolle.  Auch die Auswahl von der Unternehmensgröße (aus Kostengründen) und der Branche des Unternehmens spielt eine wichtige Rolle. Beispielsweise unterliegen Behörden oder Betrieben der Sektoren Banken, Versicherungen oder sonstigen Finanzdienstleister besonderen (Datensicherheits-)Gesetzen. Für die Einhaltung von (IT-)Compliance oder etwaigen internen Kontrollsystemen, kommt eine on-premise und in-house Lösung für die Modernisierungsmaßnahme, beziehungsweise der zukünftigen IT-Landschaft infrage. \cite{falk2012compliance}\newpage
