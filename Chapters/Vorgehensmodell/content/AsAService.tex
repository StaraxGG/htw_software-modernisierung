\section{Schichtenmodell des Cloud Computings}
\label{VgmAsAService}
%=============================

Die Art und Weise, wie eine Software über ein Cloud-System bereitgestellt wird, ist Teil der Überlegungen für die Bestimmung der Ziele und Ziellandschaft. Cloud Computing unterteile dabei in drei Schichten, die Folgendermaßen erläutert werden. Es werden die Definitionen der einzelnen Schichten von IBM verwendet \cite{ibmAsAStructre}.

\begin{description}   
 \item [\acl{IaaS}] \hfill \\
 Hier bieten die Service-Provider Cloud-Lösungen, wie etwa Speicher, Netzbetrieb oder Server bereit. Dies bietet den Unternehmen den Vorteil Kosten für den Kauf und die Wartung eigener Hardware zu sparen. Und da die Daten alle in der Cloud liegen, gibt es keinen Single point of failure. Der Single point of failure ist im Prinzip eine Fehlfunktion, die zum Ausfall des Systems führen kann. 
 \item [\acl{PaaS}]\hfill \\
 Hier wird den Usern eine komplette Cloud-Umgebung bereitgestellt. Zusätzlich zu dem Speicher und IT-Ressourcen bekommen die Nutzer eine Reihe vordefinierter Tools an die Hand. Die ist insbesondere interessant für Tests und Entwicklungen.
 \item [\acl{SaaS}] \hfill \\
Hier wird eine Anwendungen Cloud-basiert angeboten. Der Zugriff erfolgt meist über die Programmschnittstellen oder Webinterfaces. Damit kann an über, beispielsweise den Webbrowser auf Applikationen des eigenen Unternehmens entfernt zugreifen, ohne sich in einem Virtuellen privaten Netzwerk zu befinden.
 \end{description}