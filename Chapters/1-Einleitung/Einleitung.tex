\chapter{Einleitung}
Sofern eine Software in Benutzung ist, muss sie angepasst und verbessert werden. \cite{lehman_understanding_1979}\\
Dies zeigte sich bei der Einführung der \acs{DSGVO} \cite{noauthor_datenschutz-grundverordnung_nodate}. Letztere stellte unter anderem neue Anforderung an den Umgang mit personenbezogenen Daten. Bestehende Software-Systeme müssen sich diesen Anforderungen stellen. Wegen nicht-Beachtung dieser Anforderungen musste der Bekleidungshersteller H\&M im Jahr 2020 ein Bußgeld von 35,3 Millionen Euro zahlen \cite{the_hamburg_commissioner_for_data_protection_and_freedom_of_information_353_2020}.\\
Um auf äußere Einflüsse zu reagieren, muss eine Software folglich stets flexibel bleiben. Eine flexible Software kann auch an neue Anforderungen angepasst werden. Allerdings widerstehen, über viele Jahre gewachsene Software-Systeme, meist Änderungen.\\\\
Die Einführung der DSGVO zeigt anschaulich, dass diese Flexibilität einer Software essenziell für den Betrieb und Verkauf dieser ist. Neue Anforderungen können durch den Gesetzgeber formuliert werden. Anschließend muss diesen Folge geleistet werden.\\
Neben dem Gesetzgeber muss die Software auch flexibel auf die Konkurrenz reagieren können. Neue Vertriebskonzepte wie Software as a Service (SaaS) halten zunehmend mehr Einzug in der IT-Branche. Um diesen Markt bedienen zu können, muss die Software ebenfalls angepasst werden.\\
Die Frage nach der Flexibilität einer Software ist folglich nicht nebensächlich. Eine unflexible Software wird zunehmend weniger wettbewerbsfähig. Um am Markt relevant zu bleiben, muss die Flexibilität einer Software erhalten bleiben.\\\\
In dieser Arbeit werden die Gründe für eine Software-Modernisierung aufgearbeitet und konkrete Maßnahmen für eine Modernisierung der Software betrachten. 

\section{Struktur der Arbeit}
Die Arbeit ist dabei wie folgt strukturiert. Zunächst wird in Kapitel \ref{chapter2} analysiert, warum eine Software-Modernisierung benötigt wird. Kapitel \ref{chapter3} ordnet die Software-Modernisierung in den Software-Entwicklungszyklus ein. In Kapitel \ref{chapter4} wird auf das Refactoring und dessen Bedeutung für die Software-Modernisierung eingegangen. Kapitel \ref{ch:vorgehensmodell} wird das Cloud-Computing betrachten und dessen Bedeutung für die Software-Modernisierung aufzeigen. Im letzten Kapitel wird abschließend ein konkretes Vorgehensmodell für eine Modernisierung mittels des Cloud-Computings vorgestellt.